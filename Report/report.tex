% #######################################
% ########### FILL THESE IN #############
% #######################################
\def\mytitle{Coursework Report}
\def\mykeywords{Napier, SET09117, Algorithms, Data Structure, Steven, Gibson, 40270320, Steven Gibson, report}
\def\myauthor{Steven Robert Gibson}
\def\contact{40270320@live.napier.ac.uk}
\def\mymodule{Module Title (SET09117)}
% #######################################
% #### YOU DON'T NEED TO TOUCH BELOW ####
% #######################################
\documentclass[10pt, a4paper]{article}
\usepackage[a4paper,outer=1.5cm,inner=1.5cm,top=1.75cm,bottom=1.5cm]{geometry}
\twocolumn
\usepackage{graphicx}
\graphicspath{{./images/}}
%colour our links, remove weird boxes
\usepackage[colorlinks,linkcolor={black},citecolor={blue!80!black},urlcolor={blue!80!black}]{hyperref}
%Stop indentation on new paragraphs
\usepackage[parfill]{parskip}
%% Arial-like font
\IfFileExists{uarial.sty}
{
    \usepackage[english]{babel}
    \usepackage[T1]{fontenc}
    \usepackage{uarial}
    \renewcommand{\familydefault}{\sfdefault}
}{
    \GenericError{}{Couldn't find Arial font}{ you may need to install 'nonfree' fonts on your system}{}
    \usepackage{lmodern}
    \renewcommand*\familydefault{\sfdefault}
}
%Napier logo top right
\usepackage{watermark}
%Lorem Ipusm dolor please don't leave any in you final report ;)
\usepackage{lipsum}
\usepackage{xcolor}
\usepackage{listings}
%give us the Capital H that we all know and love
\usepackage{float}
%tone down the line spacing after section titles
\usepackage{titlesec}
%Cool maths printing
\usepackage{amsmath}
%PseudoCode
\usepackage{algorithm2e}

\titlespacing{\subsection}{0pt}{\parskip}{-3pt}
\titlespacing{\subsubsection}{0pt}{\parskip}{-\parskip}
\titlespacing{\paragraph}{0pt}{\parskip}{\parskip}
\newcommand{\figuremacro}[5]{
    \begin{figure}[#1]
        \centering
        \includegraphics[width=#5\columnwidth]{#2}
        \caption[#3]{\textbf{#3}#4}
        \label{fig:#2}
    \end{figure}
}

\lstset{
	escapeinside={/*@}{@*/}, language=C++,
	basicstyle=\fontsize{8.5}{12}\selectfont,
	numbers=left,numbersep=2pt,xleftmargin=2pt,frame=tb,
    columns=fullflexible,showstringspaces=false,tabsize=4,
    keepspaces=true,showtabs=false,showspaces=false,
    backgroundcolor=\color{white}, morekeywords={inline,public,
    class,private,protected,struct},captionpos=t,lineskip=-0.4em,
	aboveskip=10pt, extendedchars=true, breaklines=true,
	prebreak = \raisebox{0ex}[0ex][0ex]{\ensuremath{\hookleftarrow}},
	keywordstyle=\color[rgb]{0,0,1},
	commentstyle=\color[rgb]{0.133,0.545,0.133},
	stringstyle=\color[rgb]{0.627,0.126,0.941}
}

\thiswatermark{\centering \put(336.5,-38.0){\includegraphics[scale=0.8]{logo}} }
\title{\mytitle}
\author{\myauthor\hspace{1em}\\\contact\\Edinburgh Napier University\hspace{0.5em}-\hspace{0.5em}\mymodule}
\date{}
\hypersetup{pdfauthor=\myauthor,pdftitle=\mytitle,pdfkeywords=\mykeywords}
\sloppy
% #######################################
% ########### START FROM HERE ###########
% #######################################
\begin{document}
    \maketitle
    \begin{abstract}
The objective of this coursework was to demonstrate understanding of Algorithms and Data Structures and apply it to a real-world project. The task was to implement a game of draughts with various additional features. This was carried out successfully with a fully working game of draughts with many additional features.
    \end{abstract}
    
    
    
    \section{Introduction}
    \paragraph{Background}
The Task of the this coursework was to design and implement a game of draughts that would allow the user to play a game either player vs player or player vs computer. It also had to include an Undo/Redo feature and have a game replay feature that would replay the game autonomously. The project was written in C\# using MS Visual Studio.
    
   
	
	\section{Design}
\paragraph{}
	To design the game, a divide and conquer approach was taken, by breaking the game down into sub-problems and solving them. To help with this approach a class diagram was made(figure 1).
	\begin{figure}[H]
  	\centering
  	\includegraphics[scale = 0.4]{class-diagram}
  	\caption{Class Diagram- basic class diagram, showing the relationship between objects, most of theses are a one to one relation. Moves and Markers have a two to two relation as a the the Move class needs a before Marker and an after Marker. }
  	\label{fig:nonfloat}
	\end{figure}
To begin with, six classes were needed starting with MainWindow.cs for the GUI, a Board class to build the board, a Marker Class to make each marker, an AI Class which would be implemented once a fully functional player vs player was completed, a History for storing games and a Move class to check for valid moves. This was later changed to incorporate extra features such as the options menu and replay options window.

	\subsection{Board}
	\begin{figure}[H]
  	\centering
  	\includegraphics[scale = 0.4]{board}
  	\caption{Board and Markers - GUI of game board with markers ready to play}
  	\label{fig:nonfloat}
	\end{figure}

To begin with, the board was made using a 2D int array 8,8 so each square could be assigned to a number (-1 for invalid squares, 0 for empty squares, 1 for White markers, 2 for Black markers, 3 for White kings, 4 for Black kings) after the board was created in a text-based environment and markers were placed using the number system. It was at this point the decision to use a GUI was made because it would be a better representation for the game. The board was recreated in a GUI using stack-panels still with the same underlying numbering system. The empty board was created with buttons on the stack panels to allow the user to select them once marker are placed. The button is named with its row and column number.
	
	\subsection{Markers}
	The next stage of development focused on placing markers on the board. A case statement was used to determine the placement of the markers. This is a more effective way to carryout the placement than the original way it was done in the text-based version, which was a series of "if" statements. The markers are built using the Marker class which contains three values: the markers row and column and whether or not it has been promoted to a king. After placement is determined, the marker gets added to the corresponding stack panel for that row and column, and the button's name Gets updated with the colour of the marker - black or white. An image of the marker is placed as the background of the button to show users what each button represents. Below is an image of the standard markers.
	\begin{figure}[H]
  	\centering
  	\includegraphics[scale = 1.25]{Marker}
	\caption{Markers - game markers, White and Black Markers}
  	\label{fig:nonfloat}
	\end{figure}
	
	\subsection{Moves}
The Move class is used to create a list to check move validation and checking if a jump move is available (take an opponent's marker). Move validation is done by passing the marker colour to a method within the move class. It then checks to see if the placement chosen for the move is one column plus or minus one to the current column and plus a row for whites or minus a row for the blacks. This is done with four "if" statements. The other two check if the marker has a promotion and therefore can move up or down the rows regardless of colour. If the move is invalid a message box appears telling the user that it is invalid with a reason as to why.
	\begin{figure}[H]
  	\centering
  	\includegraphics[scale = 0.5]{errors}
	\caption{Invalid move -message box showing error when moving a white marker on a black turn}
  	\label{fig:nonfloat}
	\end{figure}
	
	
Jump validation is done using a series of "if" statements just like the move validation, but it checks two columns and plus or minus two rows depending on colour and promotion status. The jump possibilities get added to a list therefore if the list is populated a jump must occur. 
This is done by checking the number status on the board e.g. On a "White" turn, a two indicting the that square is occupied by a "Black" marker it will then check if the square behind is a zero and that option will be added to the list of jumps. An "if" statement is the used so if the list.Count() is greater than zero a jump must occur. 
  
Promotions are dealt with in the MainWindow.cs in a KingMe method. that uses two "if" statements one for if the marker is "white" and the row is equal to eight (bottom of the board for whites) or marker equal "black" and row equal to zero (top of the board for the blacks). If a marker is to be promoted the button name gets updated with the appropriate king, either "WhiteKing" or "BlackKing" and the button's background image gets updated with the appropriate king image.
\begin{figure}[H]
  	\centering
  	\includegraphics[scale = 1.25]{KingMarkers}
	\caption{Markers - game markers for promotion, White and Black King-Markers}
  	\label{fig:nonfloat}
	\end{figure} 
	\subsection{Player vs Player}
	Player vs Player (PvP) was done first to allow the game's logic to be fully tested, therefore eliminating invalid moves that a player could make, and therefore an AI player would not be-able to make as well. Making the game PvP first then adding AI later also allowed future features such as an undo feature to be added without impacting the games logic. The winning condition is calculated by counting each player's markers, and if any of the counts return a zero the other player wins.
	
	\subsection{Undo/Redo}
Once the PvP was fully working, the next stage was to get an undo feature working. To do this, a stack was created so that when a player moves a marker. The undo feature is triggered by a player using the menu or by the standard MS Windows keyboard shortcut "CTRL + Z". The undo\_stack gets added to after every move, a stack was chosen as the data structure for this as the last move in has to be the first move back out. A Deque was considered for this feature because it could also be used for the replay feature later on, however Deques are not natively supported by C\#. The undo is performed, play switches back to the other who's move was undone. If multiple undos are performed, play will switch to the player who's move was undone last, e.g. if "Black" player undoes two moves play would stay with the "Black" player if they did three moves play would return to "White" player. Originally the stack just held the position before and after into the stack in the order of the before row, before column, after row, after column. However after the redo feature was completed the stack needed to also hold the promotion status of a marker. 
\\
The redo feature was built using a stack as well. This stack gets added to only after an undo move is made to allow a player to undo the undo automatically using the menu or the standard MS Windows keyboard shortcut "CTRL + Y". It also became apparent that an extra stack would be needed for both undo and redo features to store taken markers in order to return them to the board if a move was redone after an undo had occurred.
	
	\subsection{AI}
The AI was developed using a variation of the Fisher-Yates Shuffle algorithm. In this algorithm, it puts all possibilities into a list and randomly eliminates options until only one remains. The AI class uses a variation of this in which each marker is checked to see if a zero is in an adjacent square, a returned zero indicates an empty square. It will then add that possibility to a list of moves and then a random int is taken from zero to list.Count() to make the move that matches the position within the list. However, if it finds a two indicting the that square is occupied by a "Black" marker it will then check if the square behind is a zero and take the jump, since a jump is mandatory in draughts, the same as in a PvP game.

The AI method within MainWindow.cs is an asynchronous method. This is done to give the player a better chance to see what marker the AI has moved and to give an illusion that the AI is thinking about a move. An asynchronous method had to be used instead of a standard thread.Sleep() as this would sleep the thread and not update the GUI until the very end and, lock out the user.
Below is the segment of code that delays the AI. The time is done in milliseconds and could be changed to give a more instantaneous response.

\begin{lstlisting}[caption =code to delay the AI for 1 second]
 async Task PutTaskAIDelay()
        {
            await Task.Delay(1000);
        }
\end{lstlisting}
	
	\subsection{Game Replay}
The replay Game feature was the last feature to be added. It was at this point. Additional classes were required and these were a Singleton List class, a ReplayWindow.cs Class and a Replay class. To replay, the games uses one queue and two stacks. 

A queue is used to store each of the moves just like the undo feature. As mentioned above, a deque was considered for the role. The queue allows for each move to be added one at a time by row, column, promotion status, also a first in first out which is required to show the flow of play. 

The two stacks being used are for taken markers like in the undo and redo, feature however a temporary stack is needed to flip the stack around so the taken markers are popped out in the right order. When the game board is cleared, the first stack is created and a copy of the retaken\_stack is copied into the replay taken stack

The Singleton List pattern was chosen as only one instance of a list to store previous games is required and a Singleton prevents multiple instances from being created by mistake. However, it does create a global variable, which some may view as a bad practice.

The ReplayWindow.cs (figure 6) gives the user a GUI to select which game to replay, with an option at what speed to play the game back at fast, medium or slow speed. This is done using a ComboBox. The replay comboBox is populated on the window opening and populated with gameIDs from the Singleton game list.

 \begin{figure}[H]
  	\centering
  	\includegraphics[scale = 0.5]{ReplayWindow}
	\caption{Replay - ReplayWindow.cs, User selection for game and speed of replay}
  	\label{fig:nonfloat}
	\end{figure}

The replay is an asynchronous method like the AI. The move happened instantaneously and only showed the final outcome, so an async was added to let the user see where each marker moved to and in what order. The delay is set by default to a two second delay but can be changed in the replay option window to fast (being one second), medium (being two seconds) and slow (being five seconds).

The replay class is only executed to hold a copy of the game that is currently being replayed so it can be viewed again, as when each turn is dequed from the game history it was originally lost.

\section{Enhancements}
Extra features and functionality that were added in to enhance the user experience but have no real impact on how the game runs were as follows:

An Options window to change game settings.

\begin{figure}[H]
  	\centering
  	\includegraphics[scale = 0.5]{Options}
	\caption{Options - OptionsWindow.cs, Game settings show the default settings for games}
  	\label{fig:nonfloat}
	\end{figure}

Background Music: Royalty free background music was added. As this is a personal choice for the user, the music is by default turned off but can be turned on the options menu.

Player Names: To add a more personal touch to the game, players can change the names of the players. This is just a cosmetic effect and does not affect internal logic of turns. The only difference is at the bottom the label and error messages will display the players name instead of the "Black" or "White".

\begin{figure}[H]
  	\centering
  	\includegraphics[scale = 0.75]{Winner}
	\caption{Personalisation  - Winner message,personalised Winner message}
  	\label{fig:nonfloat}
	\end{figure}

AI VS AI: This was a late addition to the game. This was added to show a new user how the game is played. This could be developed further into a full tutorial for those who don't know how to play draughts.. It was designed by taking the original AI and flipping it to the other side and have it look for spaces with 1 or 3 for jumps and minus one from rows instead of adding one to them. 

Final enchantment: A game help feature for users who are not familiar with how to draughts or checkers. This is a message box with the game's rules and winning condition. This accessible via the help menu or the standard MS Windows keyboard shortcut "F1".

\begin{figure}[H]
  	\centering
  	\includegraphics[scale = 0.6]{Help}
	\caption{Help Menu  - rules and wining condition for draughts}
  	\label{fig:nonfloat}
	\end{figure}

As stated above, the standard MS Windows keyboard shortcuts where implemented throughout the project, To give a more user friendly and professional product. 


Using the MoSCoW method of agile development, the following are extra features that would have been added:


Persistence of Data: A save game function was written and implemented, however due foreseen circumstances this feature was not developed further. The current save game option writes data to a csv file, stacks and queues within the object get written in to the csv however the csv becomes corrupted when data queues get added. The cause of this was unknown, however after some research with stackoverflow and MSDN, it does not seem possible without converting. An alternative that maybe tried in the future is to write to a json file. This was not done in this instance due to the developer's lack of knowledge of the json format at time of writing.

Multiple Jumps: This was an oversight by the developer when building the original game and did not come to light until further research into draught rules where multiple jumps are possible. An attempt was made to correct this however the implementation caused bugs to appear in certain situations. All changes were reverted back and multiple jumps are not currently possible in this build.

Custom Backgrounds: The idea of changing the colour and pattern of the game's board was thought of as another way the user could personalise their game, however due to time constraints again this idea was abandoned in favour of AI vs AI. Other cosmetic ideas were dropped as well, such as custom Markers and different audio files for background music. These cosmetic features would have been selected by the user via the options menu. 
	
	\section{Critical Evaluation}
The Game itself works without bugs and any found during testing where fixed. However, colour choice when a user selects a marker could be changed to easier see if the marker was selected or the user's pointer was just hovering. As for the code, more comments could have been added to let anyone other than the developer know how each method interacts with the game. However, each segment does have a comment to let people know what feature it deals with. Certain pieces of code could have been rearranged to flow better when reading it. Due to the way the program is structured with loops and "if" statements, performance on load has been compromised, however once loaded performance is restored and any delay is not noticeable. Also, any delay with an AI making a move was intentionally done.

	\section{Personal Evaluation}
The developer knew the time constraints and worked to them with each of the problems broken down and dealt with in a logical order. More time could have been spent with functionality instead of cosmetic details like getting persistence of data instead of optional music. One challenge that was faced during the project was in the replay function the stack was reversed, causing the game to crash. This was overcome using pen and paper to draw out the stacks and move the items in the stack between stacks. This was where the solution was found that a temporary stack was needed to flip the stack round the correct way. Another issue arose during building the AI vs AI. The "Black" player was trying to select "White" markers, which was cause by the operator being round the wrong way such that the marker row was being added instead of subtracted. To solve this, the operator was changed to a minus rather than a plus. This allowed the "Black" player to move up the board. A challenge that was not overcome was persistence of data, saving to a csv file was not possible and more research is required to save to json file. However on the whole, the project was a success as a fully playable game of draughts was built with extra functionality and extra cosmetic details.
    
   
   


		
\end{document}